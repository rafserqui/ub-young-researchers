\documentclass[11pt,xcolor={svgnames},aspectratio=169,usepdftitle=false]{beamer}

% Add space in lists
\let\toneitemize\itemize
\let\ttwoitemize\enditemize
\renewenvironment{itemize}{\toneitemize\addtolength{\itemsep}{1.05\baselineskip}}{\ttwoitemize}

\let\toneenumer\enumerate
\let\ttwoenumer\endenumerate
\renewenvironment{enumerate}{\toneenumer\addtolength{\itemsep}{1.05\baselineskip}}{\ttwoenumer}

% Itemize w bullets
\setbeamertemplate{itemize items}{$\circ$}

% Transition section slides
\setbeamertemplate{section page}
{
    \begin{centering}
    \begin{beamercolorbox}[sep=12pt,center]{part title}
        \setbeamerfont{section title}{size=\huge}
        \usebeamerfont{section title}\insertsection\par
    \end{beamercolorbox}
    \end{centering}
}

%===========================================================
% Color specifications
%===========================================================
\definecolor{GreyishBlue}{HTML}{0494e7}  % Headings color
\definecolor{CitationsBlue}{HTML}{2B6684}  % Color for citations
\definecolor{LinksPink}{HTML}{ff0054}       % Color for links

% Template for syntax highlighting
\definecolor{BeigeBackground}{HTML}{fdf6e3}
\definecolor{BlueKeyWord}{HTML}{268bd2}

\setbeamercolor{palette primary}{bg=GreyishBlue,fg=white}
\setbeamercolor{palette secondary}{bg=GreyishBlue,fg=white}
\setbeamercolor{palette tertiary}{bg=GreyishBlue,fg=white}
\setbeamercolor{palette quaternary}{bg=GreyishBlue,fg=white}
\setbeamercolor{structure}{fg=GreyishBlue} % itemize, enumerate, etc
\setbeamercolor{section in toc}{fg=GreyishBlue} % TOC sections
\setbeamercolor{background canvas}{bg=white}
\setbeamercolor{button}{bg=white, fg=GreyishBlue}
\setbeamercolor{alerted text}{fg=GreyishBlue}

\definecolor{col1}{HTML}{e71d36}
\definecolor{col2}{HTML}{2ec4b6}
\definecolor{col3}{HTML}{f77f00}

%===========================================================
% Numeration in environments
%===========================================================
\setbeamertemplate{theorems}[numbered]
\setbeamertemplate{definitions}[numbered]
\setbeamertemplate{navigation symbols}{}
\setbeamertemplate{caption}[numbered]

\usepackage{appendixnumberbeamer}

%===========================================================
% Language and font settings
%===========================================================
\usepackage[english,activeacute]{babel} % Language
\usefonttheme{professionalfonts}  % Avoid overwriting fonts

\usepackage[sfdefault,light]{FiraSans}
\usepackage[utf8]{inputenc}    % Special symbols
\usepackage[T1]{fontenc}    % T1 Encoding of font

%===========================================================
% Math font settings
%===========================================================
\usepackage{mathpazo}     % Math font
\usepackage{amsmath,amsfonts,amssymb} % Math symbols
\usepackage{dsfont}      % Math symbols like R for reals...

%===========================================================
% References' colors and PDF properties
%===========================================================
\usepackage{pgfplots}
\pgfplotsset{compat=1.15}
\usepackage{mathrsfs}
\usetikzlibrary{arrows}
\usepackage{hyperref}
\hypersetup
{
    pdfauthor={Rafael Serrano-Quintero},
    pdfsubject={UB Young Researchers 2022},
    colorlinks = {true},
    linkcolor = {GreyishBlue},
    citecolor = {LinksPink},
    urlcolor = {LinksPink},
    filecolor = {LinksPink}
}
%===========================================================
% Additional packages
%===========================================================
\usepackage{graphicx}
\usepackage{appendix}
\usepackage{marvosym}
\usepackage{enumerate} %For enumerating with letters with option [a)]
\usepackage[round]{natbib}
% Change manually the color of the parenthesis
\bibpunct{\textcolor{CitationsBlue}{(}}{\textcolor{CitationsBlue}{)}}{,}{a}{}{;}

\usepackage[flushleft]{threeparttable}
\usepackage{booktabs}
\usepackage[super]{nth}
\usepackage{float}
\usepackage{caption}
\usepackage{subcaption}
\usepackage{multicol}
\usepackage{siunitx}
\sisetup{input-open-uncertainty  = ,
         input-close-uncertainty = ,
         table-space-text-pre    = (,
         table-space-text-post   = \sym{***},
         table-align-text-pre    = false,
         table-align-text-post   = false}
\def\sym#1{\ifmmode^{#1}\else\(^{#1}\)\fi}

% For animated figure
\usepackage{animate}

\newtheorem{proposition}[theorem]{Proposition}

%==========================================================
\title{UB School of Economics Young Research Meeting}

\author{Rafael Serrano-Quintero}
\date{}

\AtBeginSection{\frame{\sectionpage}}

\defbeamertemplate{section page}{mine}[1][]{%
  \begin{centering}
    {\usebeamerfont{section name}\usebeamercolor[fg]{section name}#1}
    \vskip1em\par
    \begin{beamercolorbox}[sep=12pt,center]{part title}
      \usebeamerfont{section title}\insertsection\par
    \end{beamercolorbox}
  \end{centering}
}

\setbeamertemplate{section page}[mine]

\begin{document}

\frame{\titlepage}

\begin{frame}{Background}
    \begin{itemize}
        \item PhD Economics, 2021. \textit{University of Alicante}
        \item \alert{\textbf{Department:}} Economics
        \item \alert{\textbf{Section:}} Economic Theory
        \item \alert{\textbf{Research fields:}} Economic Growth, International Trade, Macroeconomics
        \item \alert{\textbf{Contact:}} \href{mailto:rafael.serrano@ub.edu}{rafael.serrano@ub.edu}
        \item \alert{\textbf{Web:}} \href{https://rafserqui.github.io/}{https://rafserqui.github.io/}
    \end{itemize}
\end{frame}

\section{Spatial Complementarities in Public Infrastructure and Structural Transformation in Brazil}

\begin{frame}
    \frametitle{Introduction}
    \begin{figure}[htbp]
        \begin{subfigure}[b]{0.4875\linewidth}
            \centering
                \includegraphics[height = 0.67\textheight]{../brazil-spatial-model/tex/figures/electricity-quality/dec_2010.png}
                \caption{Duration}
                \label{fig:dec_2010}
        \end{subfigure}
        \begin{subfigure}[b]{0.4875\linewidth}
            \centering
                \includegraphics[height = 0.67\textheight]{../brazil-spatial-model/tex/figures/electricity-quality/fec_2010.png}
                \caption{Frequency}
                \label{fig:fec_2010}
        \end{subfigure}
        \caption{Duration and Frequency of Power Outages}
    \end{figure}
\end{frame}

\begin{frame}
    \frametitle{Complementarities in Electricity and Roads}
    \begin{figure}
        \centering
        \animategraphics[loop,autoplay,height=0.75\textheight]{0.75}{../brazil-spatial-model/maps/animation/map-electric-roads-}{1}{6}
        \caption{Brazil Timelapse: 1960 - 2010}
    \end{figure}
\end{frame}

\begin{frame}
    \frametitle{The Paper in a Nutshell}
\alert{\textbf{What we ask:}}
\begin{itemize}
    \item Are there complementarities between quality of roads and electricity?
    \item How does it depend on structural change?
\end{itemize}
\alert{\textbf{What we do:}}
\begin{itemize}
    \item Build an economic geography model with endogenous infrastructure.
    \item Government chooses quality of roads and electricity.
    \item People react to investments.
\end{itemize}
\end{frame}

\begin{frame}
    \frametitle{Results}
\begin{itemize}
    \item We find evidence of complementarities.
    \item Welfare increases $5-16\%$.
    \item Most gains from choosing \alert{both infrastructures} optimally.
    \item Complementarities useful to $\downarrow$ agricultural employment.
    \item \alert{\textit{Some}} evidence for political alignment.
\end{itemize}
\end{frame}

\begin{frame}
    \frametitle{Punchline}

\begin{block}{}
Public infrastructures \alert{\textbf{interact}} with each other. Governments should look for these interactions and exploit them for maximum welfare increases.
\end{block}

\end{frame}

\end{document}